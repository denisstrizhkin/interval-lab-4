\documentclass[a4paper,12pt]{article}
\usepackage[utf8]{inputenc}
\usepackage[T1]{fontenc}
\usepackage{amsmath}
\usepackage{amssymb}
\usepackage[T2A]{fontenc}
\usepackage[utf8]{inputenc}
\usepackage[english,russian]{babel}
\usepackage{graphicx}
\usepackage{float}
\usepackage{listings}
\usepackage{xcolor}
\usepackage{hyperref}
\usepackage{geometry}

\geometry{top=2cm, bottom=2cm, left=2.5cm, right=1.5cm}

\definecolor{codegreen}{rgb}{0,0.6,0}
\definecolor{codegray}{rgb}{0.5,0.5,0.5}
\definecolor{codepurple}{rgb}{0.58,0,0.82}
\definecolor{backcolour}{rgb}{0.95,0.95,0.92}

\lstdefinestyle{mystyle}{
    backgroundcolor=\color{backcolour},   
    commentstyle=\color{codegreen},
    keywordstyle=\color{magenta},
    numberstyle=\tiny\color{codegray},
    stringstyle=\color{codepurple},
    basicstyle=\ttfamily\footnotesize,
    breakatwhitespace=false,         
    breaklines=true,                 
    captionpos=b,                    
    keepspaces=true,                 
    numbers=left,                    
    numbersep=5pt,                  
    showspaces=false,                
    showstringspaces=false,
    showtabs=false,                  
    tabsize=2
}
\lstset{style=mystyle}

\begin{document}

\begin{titlepage}
    \centering
    \fontsize{12pt}{14pt}\selectfont
    \textbf{Санкт-Петербургский Политехнический Университет Петра Великого}\\
    \textbf{Физико-Механический институт}\\
    \textbf{Высшая школа прикладной математики и вычислительной физики}

    \vspace{7cm}

    \fontsize{14pt}{16pt}\selectfont
    \textbf{Отчёт по лабораторной работе №4 по дисциплине}\\
    \vspace{0.5cm}
    \textbf{«Интервальный анализ»}

    \vspace{6cm}

    \begin{flushright}
        \fontsize{12pt}{14pt}\selectfont
        \begin{minipage}{0.4\textwidth}
            \raggedleft
            Выполнил:\\
            студент гр. 5040102/40201\\
            \textbf{Стрижкин Д.А.}
            
            \vspace{1cm}
            
            Проверил:\\
            доцент\\
            \textbf{Баженов А.Н.}
        \end{minipage}
    \end{flushright}

    \vfill

    \centering
    Санкт-Петербург\\
    2025
\end{titlepage}

\section{Цель работы}
Целью данной лабораторной работы является получение практических навыков вычисления интервальных описательных статистик (моды, медиан), работы с коэффициентом Жаккара и применения методов оптимизации для интервальных данных. Сравнивается эффективность различных функционалов на основе интервальных статистик для оценивания параметров моделей.

\section{Постановка задачи}
{\sloppy
Даны два бинарных файла данных диагностики томсоновского рассеяния: \texttt{-0.205\_lvl\_\allowbreak side\_\allowbreak a\_\allowbreak fast\_\allowbreak data.bin} и \texttt{0.225\_lvl\_\allowbreak side\_\allowbreak a\_\allowbreak fast\_\allowbreak data.bin}. 
\par}

Преобразование из кодов АЦП в Вольты задается формулой:
\[ V = \frac{\text{Code}}{16384} - 0.5 \]
Элементы выборки представляют собой интервалы с радиусом $r = 1/2^{14} = 1/16384$.

\subsection{Модели зависимостей}
Мы моделируем зависимость между двумя выборками $X$ и $Y$ с помощью:
\begin{enumerate}
    \item Аддитивной модели: $a + X = Y$
    \item Мультипликативной модели: $t \cdot X = Y$
\end{enumerate}

\subsection{Задачи оптимизации}
Необходимо найти параметры $a$ и $t$, максимизирующие функционалы на основе коэффициента Жаккара $F(s)$:
\begin{enumerate}
    \item $F_1(s) = J(X, Y)$ (Средний коэффициент Жаккара для исходных интервалов)
    \item $F_2(s) = J(\text{mode}(X), \text{mode}(Y))$ (По интервальной моде)
    \item $F_3(s) = J(\text{med}_K(X), \text{med}_K(Y))$ (По медиане Крейновича)
    \item $F_4(s) = J(\text{med}_P(X), \text{med}_P(Y))$ (По медиане Пролубникова)
\end{enumerate}

\section{Теоретические сведения}

\subsection{Интервальная арифметика}
Интервал определяется как $\mathbf{x} = [\underline{x}, \overline{x}]$.
\begin{itemize}
    \item Сложение: $[\underline{x}, \overline{x}] + [\underline{y}, \overline{y}] = [\underline{x} + \underline{y}, \overline{x} + \overline{y}]$
    \item Ширина: $\text{wid}(\mathbf{x}) = \overline{x} - \underline{x}$
\end{itemize}

\subsection{Коэффициент Жаккара}
Для двух интервалов $A$ и $B$ коэффициент Жаккара равен:
\[ J(A, B) = \frac{\text{wid}(A \cap B)}{\text{wid}(A \cup B)} = \frac{\min(\overline{A}, \overline{B}) - \max(\underline{A}, \underline{B})}{\max(\overline{A}, \overline{B}) - \min(\underline{A}, \underline{B})} \]

\subsection{Интервальные статистики}
\begin{itemize}
    \item \textbf{Интервальная мода}: Объединение интервалов, где достигается максимальное количество пересечений элементов выборки.
    \item \textbf{Медиана Крейновича ($\text{med}_K$)}:
    \[ \text{med}_K(X) = [\text{median}(\underline{x}_i), \text{median}(\overline{x}_i)] \]
    \item \textbf{Медиана Пролубникова ($\text{med}_P$)}: Вычисляется на основе центральных элементов вариационного ряда (упорядоченного по серединам интервалов).
\end{itemize}

\subsection{Оптимизация}
Используется \textbf{метод золотого сечения} для поиска $s_{max} = \arg \max F(s)$ с точностью сходимости $\epsilon = 5 \times 10^{-4}$.

\section{Реализация}

Решение реализовано на языке \textbf{Rust}. 

\subsection{Обработка данных}
Бинарные файлы содержат глобальный заголовок (256 байт), за которым следуют кадры данных. Каждый кадр (16400 байт) имеет 16-байтовый заголовок. Точки данных представляют собой 16-битные целые числа (Little Endian). Программа итерируется по файлу, пропуская заголовки, применяет маску 14 бит (`0x3FFF`) и преобразует значения в Вольты.

\subsection{Структура кода}
Структура \texttt{Interval} реализует арифметические операции и вычисление коэффициента Жаккара.
Функция \texttt{interval\_mode} реализует алгоритм "sweep-line" для нахождения области максимального перекрытия.
Функции \texttt{med\_k} и \texttt{med\_p} сортируют границы или интервалы соответственно для нахождения устойчивых центральных тенденций.

\begin{lstlisting}[language=C, caption=Реализация метода золотого сечения]
pub fn golden_section_search<F>(mut f: F, mut a: f64, mut b: f64, tol: f64) -> f64 
where F: FnMut(f64) -> f64 {
    let phi = (5.0_f64.sqrt() - 1.0) / 2.0;
    let mut c = b - phi * (b - a);
    let mut d = a + phi * (b - a);
    // ... loop until (b - a) < tol
    (a + b) / 2.0
}
\end{lstlisting}

\section{Результаты}

Оптимизация выполнена с точностью $\epsilon = 5 \times 10^{-4}$. Вычисленные параметры сравниваются с контрольными значениями.

\begin{table}[H]
\centering
\caption{Результаты оценки параметров $a$ и $t$ для различных функционалов $J_i$}
\begin{tabular}{|c|c|c|}
\hline
\textbf{const} & \textbf{Оценка} & \textbf{Значение функционала} \\
\hline
\multicolumn{3}{|c|}{$J_i(a/t, X, Y)$} \\
\hline
$\hat{a}$ & $0.3408 \pm 0.0005$ & $-0.7857$ \\
\hline
$\hat{t}$ & $-1.0519 \pm 0.0005$ & $-0.8617$ \\
\hline
\multicolumn{3}{|c|}{$J_i(a/t, \text{mode}X, \text{mode}Y)$} \\
\hline
$\hat{a}$ & $0.3469 \pm 0.0005$ & $0.1632$ \\
\hline
$\hat{t}$ & $-1.0396 \pm 0.0005$ & $0.2405$ \\
\hline
\multicolumn{3}{|c|}{$J_i(a/t, \text{med}_K X, \text{med}_K Y)$} \\
\hline
$\hat{a}$ & $0.3442 \pm 0.0005$ & $0.9622$ \\
\hline
$\hat{t}$ & $-1.0278 \pm 0.0005$ & $0.9730$ \\
\hline
\multicolumn{3}{|c|}{$J_i(a/t, \text{med}_P X, \text{med}_P Y)$} \\
\hline
$\hat{a}$ & $0.3442 \pm 0.0005$ & $0.9622$ \\
\hline
$\hat{t}$ & $-1.0278 \pm 0.0005$ & $0.9730$ \\
\hline
\end{tabular}
\end{table}

\subsection{Графики}

\begin{figure}[H]
    \centering
    \includegraphics[width=0.45\textwidth]{images/f1_a.png}
    \includegraphics[width=0.45\textwidth]{images/f1_t.png}
    \caption{Функционал $F_1$ (Raw) от параметров $a$ (слева) и $t$ (справа)}
\end{figure}

\begin{figure}[H]
    \centering
    \includegraphics[width=0.45\textwidth]{images/f2_a.png}
    \includegraphics[width=0.45\textwidth]{images/f2_t.png}
    \caption{Функционал $F_2$ (Mode) от параметров $a$ (слева) и $t$ (справа)}
\end{figure}

\begin{figure}[H]
    \centering
    \includegraphics[width=0.45\textwidth]{images/f3_a.png}
    \includegraphics[width=0.45\textwidth]{images/f3_t.png}
    \caption{Функционал $F_3$ (MedK) от параметров $a$ (слева) и $t$ (справа)}
\end{figure}

\begin{figure}[H]
    \centering
    \includegraphics[width=0.45\textwidth]{images/f4_a.png}
    \includegraphics[width=0.45\textwidth]{images/f4_t.png}
    \caption{Функционал $F_4$ (MedP) от параметров $a$ (слева) и $t$ (справа)}
\end{figure}

\subsection{Сравнение результатов}
Сравнение результатов, полученных методом золотого сечения (табл. 1), с эталонными значениями показывает, что различия между ними незначительны и не превышают $1 \times 10^{-3}$ по каждому из параметров $a^*$ и $t^*$.

\begin{itemize}
    \item \textbf{Функционал $J_1(a/t, X, Y)$.} Различие между вычисленными и фактическими значениями составляет: $\Delta a = |0.3408 - 0.3409| = 0.0001$, $\Delta t = |-1.0519 - (-1.0509)| = 0.0010$. Оба значения лежат в пределах допустимой погрешности. Значения функционала отличаются незначительно, что подтверждает корректность метода.
    
    \item \textbf{Функционал $J_2(a/t, \text{mode}X, \text{mode}Y)$.} Вычисленные параметры: $a^* = 0.3469$, $t^* = -1.0396$; фактические: $a^* = 0.3468$, $t^* = -1.0391$. Расхождения $\Delta a = 0.0001$ и $\Delta t = 0.0005$ незначительны, однако наблюдается высокая чувствительность функционала в окрестности экстремума, что связано с узким "игольчатым" максимумом функции.
    
    \item \textbf{Функционал $J_3(a/t, \text{med}_K X, \text{med}_K Y)$.} Параметры, найденные методом золотого сечения, близки к фактическим: $a^*_{\text{calc}} = 0.3442$, $a^*_{\text{fact}} = 0.3444$, $t^*_{\text{calc}} = -1.0278$, $t^*_{\text{fact}} = -1.0272$. Различия по аргументам не превышают $6 \times 10^{-4}$.
    
    \item \textbf{Функционал $J_4(a/t, \text{med}_P X, \text{med}_P Y)$.} Параметры также практически совпадают с эталонными значениями (различие $< 6 \times 10^{-4}$), что подтверждает устойчивость медианных оценок.
\end{itemize}

\subsection{Вывод}
В ходе лабораторной работы были реализованы методы оценки параметров $a$ и $t$ для различных функционалов Жаккара $J_1-J_4$ на основе интервальных статистик.
Метод золотого сечения показал высокую точность: расхождения между вычисленными и фактическими значениями параметров не превышают $1 \times 10^{-3}$, что подтверждает корректность реализации и сходимость процедуры оптимизации.

Анализ функционалов показал следующие особенности:
\begin{itemize}
    \item Для исходных данных ($J_1$) значения функционала отрицательные, что указывает на низкую степень совпадения между выборками в среднем.
    \item При использовании модальных оценок ($J_2$) точность остаётся высокой, однако вычисления требуют тщательного выбора шага поиска из-за узкого профиля экстремума.
    \item Компонентные и парные медианные оценки ($J_3, J_4$) обеспечивают более гладкий профиль функции, устойчивое положение экстремума и наибольшие значения функционала (до 0.96-0.97), что свидетельствует о повышении точности и стабильности метода.
\end{itemize}

Таким образом, медианные методы ($J_3, J_4$) оказались наиболее эффективными с точки зрения устойчивости и воспроизводимости результатов, в то время как модальный подход ($J_2$) характеризуется повышенными вычислительными требованиями. Метод золотого сечения подтвердил свою применимость для поиска экстремумов в задачах интервального анализа.

\end{document}
