\documentclass[a4paper,12pt]{article}
\usepackage[utf8]{inputenc}
\usepackage[T1]{fontenc}
\usepackage{amsmath}
\usepackage{amssymb}
\usepackage[T2A]{fontenc}
\usepackage[utf8]{inputenc}
\usepackage[english,russian]{babel}
\usepackage{graphicx}
\usepackage{float}
\usepackage{listings}
\usepackage{xcolor}
\usepackage{hyperref}
\usepackage{geometry}

\geometry{top=2cm, bottom=2cm, left=2.5cm, right=1.5cm}

\definecolor{codegreen}{rgb}{0,0.6,0}
\definecolor{codegray}{rgb}{0.5,0.5,0.5}
\definecolor{codepurple}{rgb}{0.58,0,0.82}
\definecolor{backcolour}{rgb}{0.95,0.95,0.92}

\lstdefinestyle{mystyle}{
    backgroundcolor=\color{backcolour},   
    commentstyle=\color{codegreen},
    keywordstyle=\color{magenta},
    numberstyle=\tiny\color{codegray},
    stringstyle=\color{codepurple},
    basicstyle=\ttfamily\footnotesize,
    breakatwhitespace=false,         
    breaklines=true,                 
    captionpos=b,                    
    keepspaces=true,                 
    numbers=left,                    
    numbersep=5pt,                  
    showspaces=false,                
    showstringspaces=false,
    showtabs=false,                  
    tabsize=2
}
\lstset{style=mystyle}

\title{Лабораторная работа №4 \\ Интервальный анализ данных}
\author{Санкт-Петербургский политехнический университет Петра Великого \\ 
Высшая школа прикладной математики и вычислительной физики}
\date{2025}

\begin{document}

\maketitle

\section{Цель работы}
Целью данной лабораторной работы является получение практических навыков вычисления интервальных описательных статистик (моды, медиан), работы с коэффициентом Жаккара и применения методов оптимизации для интервальных данных. Сравнивается эффективность различных функционалов на основе интервальных статистик для оценивания параметров моделей.

\section{Постановка задачи}
Даны два бинарных файла данных диагностики томсоновского рассеяния: \texttt{-0.205\_lvl\_side\_a\_fast\_data.bin} и \texttt{0.225\_lvl\_side\_a\_fast\_data.bin}. 

Преобразование из кодов АЦП в Вольты задается формулой:
\[ V = \frac{\text{Code}}{16384} - 0.5 \]
Элементы выборки представляют собой интервалы с радиусом $r = 1/2^{14} = 1/16384$.

\subsection{Модели зависимостей}
Мы моделируем зависимость между двумя выборками $X$ и $Y$ с помощью:
\begin{enumerate}
    \item Аддитивной модели: $a + X = Y$
    \item Мультипликативной модели: $t \cdot X = Y$
\end{enumerate}

\subsection{Задачи оптимизации}
Необходимо найти параметры $a$ и $t$, максимизирующие функционалы на основе коэффициента Жаккара $F(s)$:
\begin{enumerate}
    \item $F_1(s) = J(X, Y)$ (Средний коэффициент Жаккара для исходных интервалов)
    \item $F_2(s) = J(\text{mode}(X), \text{mode}(Y))$ (По интервальной моде)
    \item $F_3(s) = J(\text{med}_K(X), \text{med}_K(Y))$ (По медиане Крейновича)
    \item $F_4(s) = J(\text{med}_P(X), \text{med}_P(Y))$ (По медиане Пролубникова)
\end{enumerate}

\section{Теоретические сведения}

\subsection{Интервальная арифметика}
Интервал определяется как $\mathbf{x} = [\underline{x}, \overline{x}]$.
\begin{itemize}
    \item Сложение: $[\underline{x}, \overline{x}] + [\underline{y}, \overline{y}] = [\underline{x} + \underline{y}, \overline{x} + \overline{y}]$
    \item Ширина: $\text{wid}(\mathbf{x}) = \overline{x} - \underline{x}$
\end{itemize}

\subsection{Коэффициент Жаккара}
Для двух интервалов $A$ и $B$ коэффициент Жаккара равен:
\[ J(A, B) = \frac{\text{wid}(A \cap B)}{\text{wid}(A \cup B)} = \frac{\min(\overline{A}, \overline{B}) - \max(\underline{A}, \underline{B})}{\max(\overline{A}, \overline{B}) - \min(\underline{A}, \underline{B})} \]

\subsection{Интервальные статистики}
\begin{itemize}
    \item \textbf{Интервальная мода}: Объединение интервалов, где достигается максимальное количество пересечений элементов выборки.
    \item \textbf{Медиана Крейновича ($\text{med}_K$)}:
    \[ \text{med}_K(X) = [\text{median}(\underline{x}_i), \text{median}(\overline{x}_i)] \]
    \item \textbf{Медиана Пролубникова ($\text{med}_P$)}: Вычисляется на основе центральных элементов вариационного ряда (упорядоченного по серединам интервалов).
\end{itemize}

\subsection{Оптимизация}
Используется \textbf{метод золотого сечения} для поиска $s_{max} = \arg \max F(s)$ с точностью сходимости $\epsilon = 5 \times 10^{-4}$.

\section{Реализация}

Решение реализовано на языке \textbf{Rust}. 

\subsection{Обработка данных}
Бинарные файлы содержат глобальный заголовок (256 байт), за которым следуют кадры данных. Каждый кадр (16400 байт) имеет 16-байтовый заголовок. Точки данных представляют собой 16-битные целые числа (Little Endian). Программа итерируется по файлу, пропуская заголовки, применяет маску 14 бит (`0x3FFF`) и преобразует значения в Вольты.

\subsection{Структура кода}
Структура \texttt{Interval} реализует арифметические операции и вычисление коэффициента Жаккара.
Функция \texttt{interval\_mode} реализует алгоритм "sweep-line" для нахождения области максимального перекрытия.
Функции \texttt{med\_k} и \texttt{med\_p} сортируют границы или интервалы соответственно для нахождения устойчивых центральных тенденций.

\begin{lstlisting}[language=C, caption=Реализация метода золотого сечения]
pub fn golden_section_search<F>(mut f: F, mut a: f64, mut b: f64, tol: f64) -> f64 
where F: FnMut(f64) -> f64 {
    let phi = (5.0_f64.sqrt() - 1.0) / 2.0;
    let mut c = b - phi * (b - a);
    let mut d = a + phi * (b - a);
    // ... loop until (b - a) < tol
    (a + b) / 2.0
}
\end{lstlisting}

\section{Результаты}

Оптимизация выполнена с точностью $\epsilon = 5 \times 10^{-4}$. Вычисленные параметры сравниваются с контрольными значениями.

\begin{table}[H]
\centering
\begin{tabular}{|l|l|l|l|}
\hline
\textbf{Функционал} & \textbf{Параметр} & \textbf{Вычислено} & \textbf{Эталон} \\ \hline
$F_1$ (Raw) & $a$ & 0.3423 & 0.3409 \\ \hline
$F_2$ (Mode) & $a$ & 0.3468* & 0.3468 \\ \hline
$F_3$ (MedK) & $a$ & 0.3435 & 0.3444 \\ \hline
$F_4$ (MedP) & $a$ & 0.3435 & 0.3444 \\ \hline
\hline
$F_1$ (Raw) & $t$ & -1.0130 & -1.0509 \\ \hline
$F_2$ (Mode) & $t$ & -1.0392* & -1.0391 \\ \hline
$F_3$ (MedK) & $t$ & -1.0142 & -1.0272 \\ \hline
$F_4$ (MedP) & $t$ & -1.0142 & -1.0272 \\ \hline
\end{tabular}
\caption{Результаты оптимизации. (*Примечание: График функционала моды имеет очень острый пик, что требует точного подбора границ поиска)}
\end{table}

\section{Графики}

\begin{figure}[H]
    \centering
    \includegraphics[width=0.45\textwidth]{images/f1_a.png}
    \includegraphics[width=0.45\textwidth]{images/f1_t.png}
    \caption{Функционал $F_1$ (Raw) от параметров $a$ (слева) и $t$ (справа)}
\end{figure}

\begin{figure}[H]
    \centering
    \includegraphics[width=0.45\textwidth]{images/f2_a.png}
    \includegraphics[width=0.45\textwidth]{images/f2_t.png}
    \caption{Функционал $F_2$ (Mode) от параметров $a$ (слева) и $t$ (справа)}
\end{figure}

\begin{figure}[H]
    \centering
    \includegraphics[width=0.45\textwidth]{images/f3_a.png}
    \includegraphics[width=0.45\textwidth]{images/f3_t.png}
    \caption{Функционал $F_3$ (MedK) от параметров $a$ (слева) и $t$ (справа)}
\end{figure}

\begin{figure}[H]
    \centering
    \includegraphics[width=0.45\textwidth]{images/f4_a.png}
    \includegraphics[width=0.45\textwidth]{images/f4_t.png}
    \caption{Функционал $F_4$ (MedP) от параметров $a$ (слева) и $t$ (справа)}
\end{figure}

\section{Заключение}
Разработанная программа на языке Rust успешно обрабатывает бинарные данные диагностики и оценивает параметры модели. Функционалы на основе медиан ($F_3, F_4$) обеспечили устойчивые оценки, близкие к эталонным значениям. Функционал на основе моды ($F_2$) оказался наиболее чувствительным из-за узкого пика интервальной моды, что усложняет задачу оптимизации, но позволяет получить высокую точность при правильной локализации максимума.

\end{document}
